%% This file is part of the spectroscopic_parallax project
%% Copyright 2018 the authors.

% To-do items
% -----------
% - 

\documentclass[modern]{aastex62}
\usepackage{amsmath}

%% Reintroduced the \received and \accepted commands from AASTeX v5.2
%\received{TBD}
%\revised{TBD}
%\accepted{TBD}
%% Command to document which AAS Journal the manuscript was submitted to.
%% Adds "Submitted to " the arguement.
%\submitjournal{AAS Journals}

\shorttitle{kinematic distance to the galactic center}
\shortauthors{hogg, eilers, rix}
%\watermark{kittens}

% project macros
\newcommand{\acronym}[1]{{\small{#1}}}
\newcommand{\project}[1]{\textsl{#1}}
\newcommand{\apogee}{\project{\acronym{APOGEE}}}
\newcommand{\gaia}{\project{Gaia}}
\newcommand{\wise}{\project{\acronym{WISE}}}
\newcommand{\zmass}{\project{\acronym{2MASS}}}

% jargon macros
\newcommand{\sagastar}{Sag\,A$^{\ast}$}

% math macros
\newcommand{\logg}{\log g}
\DeclareMathOperator*{\argmin}{argmin}

% tweak the modern style -- trust in Hogg
\setlength{\parindent}{1.0\baselineskip}
\addtolength{\textheight}{0.5in}
\addtolength{\topmargin}{-0.3in}

\begin{document}\sloppy\sloppypar\raggedbottom\frenchspacing % trust in Hogg

\title{\textbf{%
A purely kinematic distance to the Galactic Center
}}

\author[0000-0003-2866-9403]{David W. Hogg}
\affiliation{Center for Cosmology and Particle Physics, Department of Physics, New York University}
\affiliation{Center for Data Science, New York University}
\affiliation{Max-Planck-Institut f\"ur Astronomie}
\affiliation{Flatiron Institute}

\author[0000-0003-2895-6218]{Anna-Christina Eilers}
\affiliation{Max-Planck-Institut f\"ur Astronomie}

\author[0000-0003-4996-9069]{Hans-Walter Rix}
\affiliation{Max-Planck-Institut f\"ur Astronomie}

\begin{abstract}\noindent
% Context
The distance from the Sun to the Galactic Center is a critical parameter
that sets the length scale for many Galactic measurements and phenomena.
Radio observations have delivered an extremely precise proper motion
of the Center (or really \sagastar),
but not a distance with comparable precision.
Meanwhile, the \gaia\ Mission is delivering distances and proper motions for
billions of stars, some of which extend to $>10$~kpc.
% Aims
Here we produce a purely kinematic measurement of the position of the
Galactic Center.
% Methods
We use the \gaia\ data to calibrate accurate but simple data-driven
spectrophotometric distance estimates to luminous red-giant stars.
We compare the mean proper motions of these stars as a function of distance
and azimuth in the Milky-Way disk to the azimuth and proper motion of
\sagastar.
% Results
We find a kinematic distance to the Galactic Center of $XXX\pm YYY$~kpc.
This distance measurement is purely kinematic and geometric;
it involves no physical or dynamical model of either stars or the Milky Way.
\end{abstract}

\keywords{%
Galaxy:~center
 ---
Galaxy:~fundamental~parameters
 ---
Galaxy:~kinematics~and~dynamics
 ---
infrared:~stars
 ---
%methods:~data~analysis
% ---
proper~motions
 ---
stars:~distances
 ---
stars:~kinematics~and~dynamics
% ---
%techniques:~spectroscopic
}

\section{Introduction} \label{sec:intro}

\section{Assumptions of the method}

Our position is that a methodological technique is correct inasmuch as
it delivers correct or best results \emph{under its particular assumptions}.
In that spirit, we present here the assumptions of the method
explicitly.
We will return to these assumptions in the Discussion section below,
to address the costs and benefits of each in more depth.
\begin{description}
\item[features] Wassuh!

\end{description}

\section{Method}

\section{Data}

\section{Results and validation}

\section{Discussion}

\acknowledgements
It is a pleasure to thank
  WHOEVER
for help with this project.

This work has made use of data from the European Space Agency (ESA) mission
\gaia\ (\url{https://www.cosmos.esa.int/gaia}), processed by the \gaia\ Data
Processing and Analysis Consortium (\acronym{DPAC},
\url{https://www.cosmos.esa.int/web/gaia/dpac/consortium}). Funding for the
\acronym{DPAC}
has been provided by national institutions, in particular the institutions
participating in the \gaia\ Multilateral Agreement.

\begin{thebibliography}{}
\bibitem[Astropy Collaboration et al.(2013)]{2013A&A...558A..33A} Astropy Collaboration, Robitaille, T.~P., Tollerud, E.~J., et al.\ 2013, \aap, 558, A33 
\bibitem[Bertin \& Arnouts(1996)]{1996A&AS..117..393B} Bertin, E., \& Arnouts, S.\ 1996, \aaps, 117, 393 
\bibitem[Corrales(2015)]{2015ApJ...805...23C} Corrales, L.\ 2015, \apj, 805, 23
\bibitem[Ferland et al.(2013)]{2013RMxAA..49..137F} Ferland, G.~J., Porter, R.~L., van Hoof, P.~A.~M., et al.\ 2013, \rmxaa, 49, 137
\bibitem[Hanisch \& Biemesderfer(1989)]{1989BAAS...21..780H} Hanisch, R.~J., \& Biemesderfer, C.~D.\ 1989, \baas, 21, 780 
\bibitem[Lamport(1994)]{lamport94} Lamport, L. 1994, LaTeX: A Document Preparation System, 2nd Edition (Boston, Addison-Wesley Professional)
\bibitem[Schwarz et al.(2011)]{2011ApJS..197...31S} Schwarz, G.~J., Ness, J.-U., Osborne, J.~P., et al.\ 2011, \apjs, 197, 31  
\bibitem[Vogt et al.(2014)]{2014ApJ...793..127V} Vogt, F.~P.~A., Dopita, M.~A., Kewley, L.~J., et al.\ 2014, \apj, 793, 127  
\end{thebibliography}

\end{document}
