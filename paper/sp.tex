%% This file is part of the spectroscopic_parallax project
%% Copyright 2018 the authors.

\documentclass[modern]{aastex62}

%% Reintroduced the \received and \accepted commands from AASTeX v5.2
%\received{TBD}
%\revised{TBD}
%\accepted{TBD}
%% Command to document which AAS Journal the manuscript was submitted to.
%% Adds "Submitted to " the arguement.
%\submitjournal{AAS Journals}

\shorttitle{spectroscopic distance estimates}
\shortauthors{eilers, hogg, rix, ness}
%\watermark{kittens}

% project macros
\newcommand{\acronym}[1]{{\small{#1}}}
\newcommand{\project}[1]{\textsl{#1}}
\newcommand{\apogee}{\project{\acronym{APOGEE}}}
\newcommand{\gaia}{\project{Gaia}}
\newcommand{\wise}{\project{\acronym{WISE}}}
\newcommand{\zmass}{\project{\acronym{2MASS}}}
\newcommand{\sdssv}{\project{\acronym{SDSS-V}}}

% math macros
\newcommand{\logg}{\log g}

\begin{document}\sloppy\sloppypar\raggedbottom\frenchspacing

\title{Accurate spectroscopic estimates of luminosity and distance\\
       for luminous red-giant stars}

\correspondingauthor{Anna-Christina Eilers}
\email{eilers@mpia.de}

\author[0000-0003-2895-6218]{Anna-Christina Eilers}
\affiliation{Max-Planck-Institut f\"ur Astronomie}

\author[0000-0003-2866-9403]{David W. Hogg}
\affiliation{Max-Planck-Institut f\"ur Astronomie}
\affiliation{Center for Cosmology and Particle Physics, Department of Physics, New York University}
\affiliation{Center for Data Science, New York University}
\affiliation{Flatiron Institute}

\author{Hans-Walter Rix}
\affiliation{Max-Planck-Institut f\"ur Astronomie}

\author{Melissa Ness}
\affiliation{Flatiron Institute}
\affiliation{Department of Astronomy, Columbia University}

\begin{abstract}\noindent
With contemporary infrared spectroscopic surveys like \apogee,
red-giant stars can be observed to distances and extinctions
at which geometric parallaxes are not precisely measured by \gaia.
Here we use the \apogee--\gaia\ overlap to train a purely linear model of
continuum-normalized \apogee\ spectroscopy
(plus \gaia, \zmass, and \wise\ photometry)
that predicts inverse-square-root luminosity for red stars with $0<\logg<2.2$.
The inverse-square-root is used to make the noise model close to Gaussian,
so that the training can be performed with full use of uncertainties and
without any cuts on parallax or parallax signal-to-noise ratio;
the model includes an L1 regularization that zeros out the contributions of
most spectral pixels to the parallax estimates; and the training was performed
with leave-out subsamples so no star's astrometry is used even indirectly in its
spectroscopic parallax estimate.
The model is flexible enough to correct for extinction by dust without any
input of explicit extinction or reddening estimates.
We re-label each star in the sample
with a new spectroscopic parallax; this new parallax has an empirical uncertainty of
10\,percent, which is more precise than the \gaia\ parallax
for the vast majority of targets.
Validation with globular and open clusters shows that the spectroscopic parallaxes
are both accurate and precise.
These distance and luminosity estimates open up great opportunities for
mapping the Milky Way disk with the next
generation of spectroscopic surveys, and especially \sdssv.
\end{abstract}

\keywords{%
methods:~statistical
 ---
techniques:~spectroscopic
 ---
catalogs
 ---
surveys
 ---
parallaxes
 ---
stars:~distances
 ---
Galaxy:~disk
 ---
infrared:~stars
}

\section{Introduction} \label{sec:intro}

foo

bar

\section{Data}

foo

bar

\section{Method}

\section{Validation}

\section{Results and Discussion}

\acknowledgements
This project was developed in part at the
2018 \acronym{NYC} Gaia Sprint, hosted by the Center for Computational Astrophysics of
the Flatiron Institute in New York City.

This work has made use of data from the European Space Agency (ESA) mission
\gaia\ (\url{https://www.cosmos.esa.int/gaia}), processed by the \gaia\ Data
Processing and Analysis Consortium (\acronym{DPAC},
\url{https://www.cosmos.esa.int/web/gaia/dpac/consortium}). Funding for the
\acronym{DPAC}
has been provided by national institutions, in particular the institutions
participating in the \gaia\ Multilateral Agreement.

\begin{thebibliography}{}

\bibitem[Astropy Collaboration et al.(2013)]{2013A&A...558A..33A} Astropy Collaboration, Robitaille, T.~P., Tollerud, E.~J., et al.\ 2013, \aap, 558, A33 
\bibitem[Bertin \& Arnouts(1996)]{1996A&AS..117..393B} Bertin, E., \& Arnouts, S.\ 1996, \aaps, 117, 393 
\bibitem[Corrales(2015)]{2015ApJ...805...23C} Corrales, L.\ 2015, \apj, 805, 23
\bibitem[Ferland et al.(2013)]{2013RMxAA..49..137F} Ferland, G.~J., Porter, R.~L., van Hoof, P.~A.~M., et al.\ 2013, \rmxaa, 49, 137
\bibitem[Hanisch \& Biemesderfer(1989)]{1989BAAS...21..780H} Hanisch, R.~J., \& Biemesderfer, C.~D.\ 1989, \baas, 21, 780 
\bibitem[Lamport(1994)]{lamport94} Lamport, L. 1994, LaTeX: A Document Preparation System, 2nd Edition (Boston, Addison-Wesley Professional)
\bibitem[Schwarz et al.(2011)]{2011ApJS..197...31S} Schwarz, G.~J., Ness, J.-U., Osborne, J.~P., et al.\ 2011, \apjs, 197, 31  
\bibitem[Vogt et al.(2014)]{2014ApJ...793..127V} Vogt, F.~P.~A., Dopita, M.~A., Kewley, L.~J., et al.\ 2014, \apj, 793, 127  

\end{thebibliography}

%% This command is needed to show the entire author+affilation list when
%% the collaboration and author truncation commands are used.  It has to
%% go at the end of the manuscript.
%\allauthors

%% Include this line if you are using the \added, \replaced, \deleted
%% commands to see a summary list of all changes at the end of the article.
%\listofchanges

\end{document}

% End of file `sample62.tex'.
